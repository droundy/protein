\documentclass[letterpaper,twocolumn,amsmath,amssymb,pre]{revtex4-1}
\usepackage{graphicx}% Include figure files
\usepackage{dcolumn}% Align table columns on decimal point
\usepackage{bm}% bold math
\usepackage{color}
\usepackage{breqn}

\newcommand{\red}[1]{{\bf \color{red} #1}}
\newcommand{\blue}[1]{{\bf \color{blue} #1}}
\newcommand{\green}[1]{{\bf \color{green} #1}}
\newcommand{\rr}{\textbf{r}}
\newcommand{\refnote}{\red{[ref]}}

\newcommand{\fixme}[1]{\red{[#1]}}

%\newcommand{\derivation}[1]{#1} % Use this to show all derivations in detail
\newcommand{\derivation}[1]{} % Use this for nice pegagogical paper...

% needsworklater is used to annotate bits that need work, but that we
% can postpone for a while.
\newcommand{\needsworklater}[1]{\emph{[#1]}}
% needsworknow is intended to prioritize stuff that needs fixing.
\newcommand{\needsworknow}[1]{\textcolor{red}{[\emph{#1}]}}

\begin{document}
\title{E Coli project paper}

%\pacs{61.20.Ne, 61.20.Gy, 61.20.Ja}
%%%%%%%%%%%%%%%%%%%%%%%%%%%%%%%%%%%%%%%%%%%%%%%%%%%%%%%%%%%%
\begin{abstract}
  This paper is about science.
\end{abstract}

\section{Why does the E. Coli distribution matter?}
\subsection{E. Coli are important}
\subsection{E. Coli theory enables treatment of inhomogeneous systems}
\subsection{E. Coli distribution function is key input in thermodynamic perturbation theory}
\subsection{Explain what the $a_1$ in thermodynamic E. Coli theory is}

\section{What E. Coli has been done before?}
\subsection{Gloor simple method, problematic E. Coli}

\subsection{Fischer accurate but E. Coli}

\section{Triplet E. Coli correlation function?}

\section{What is our new E. Coli?}

\section{What does E. Coli look like (graphs that show the E. Coli itself)?}
\subsection{Fit to radial E. Coli function}

\section{How well does E. Coli work?}

\section{THE END OF THE E. COLI}

%%%%%%%%%%%%%%%%%%%%%%%%%%%%%%%%%%%%%%%%%%%%%%%%%%%%%%%%%%%%
\section{Introduction to E. Coli}

\section{Previous E. Coli}

\section{Pair distribution function with E. Coli}

\section{Goals of this E. Coli (to be integrated into rest of paper)}

\section{Theoretical E. Coli}

\section{Comparison with e. coli}\label{sec:comparison}

\section{Conclusion}

\appendix

\section*{Appendix}
%\cite{jin2011perturbative} Combines hard sphere (white bear) with attraction that is not mean field but rather uses two other correltation expressions given in  -\cite{tang2008accurate} and \cite{hlushak2009direct}

%Tang \cite{tang2008accurate}Derives the direct correltation for a SW and Yukawa and LJ potentials by first getting the radial distribution function through a method I don't really understand.  Composes fourier transforms $c(k)$ and $h(k)$ of these complicated functions that depend on filling fraction and the potential in an integral. Uses first order mean spherical approximation.  This is just square potential well around particle but the radial distribution function takes only the first order in an expansion in... (high temp?).

%\bibliography{paper}% Produces the bibliography via BibTeX.


\end{document}
