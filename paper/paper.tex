\documentclass[letterpaper,twocolumn,amsmath,amssymb,pre]{revtex4-1}
\usepackage{graphicx}% Include figure files
\usepackage{color}

\newcommand{\red}[1]{{\bf \color{red} #1}}
\newcommand{\blue}[1]{{\bf \color{blue} #1}}
\newcommand{\green}[1]{{\bf \color{green} #1}}

\newcommand{\fixme}[1]{\red{[#1]}}
\newcommand{\davidsays}[1]{{\color{red} [\green{David:} \emph{#1}]}}
\newcommand{\renesays}[1]{{\color{red} [\blue{Rene:} \emph{#1}]}}
\newcommand{\jeffsays}[1]{{\color{red} [\blue{Jeff:} \emph{#1}]}}

\newcommand\micron{\ensuremath{\mu\text{m}}}

\begin{document}
\title{Robustness of MinD oscillation in \emph{Escherichia coli} with
  diverse cell shapes}

\author{Jeff B. Schulte}
\affiliation{Department of Physics, Oregon State University}
\author{Rene W. Zeto}
\affiliation{Department of Physics, Oregon State University}
\author{David Roundy}
\affiliation{Department of Physics, Oregon State University}

\begin{abstract}
  The dynamics of the Min-protein system help Escherichia coli
  bacteria regulate the process of cell division by identifying the
  center of the cell.  We model the Min-protein system in bacteria
  that have been forced into unusual flattened shapes, as have
  recently been experimentally observed.  We find that although the
  presence of Min oscillations is robust in a wide variety of cellular
  configurations, the location of the peaks is strongly affected by
  the cellular shape.  In some cases no periodic oscillations are
  observed.  In particular, we find that cellular shapes observed
  experimentally to present irregular oscillations do so in the
  theoretical model, consistently \fixme{or inconsistently?} with
  experiment.  \fixme{In agreement with previous theoretical and
    experimental results, we observe ``rotating'' behavior in certain
    shapes having three corners.}
\end{abstract}

\maketitle

\section{Introduction}
It is vital that during the process of bacterial cell division a cell
avoid minicelling, or splitting into daughter cells with lopsided
volumes.  Instrumental to this process in \emph{Escherichia coli} is a
long FtsZ polymer chain that develops on the cell wall in the center
region of the cell, helping dictate the plane of
division~\cite{adams2009bacterial, lutkenhaus2007assembly}. Previous
experimental studies have shown that the MinC protein, known to
inhibit the FtZ polymer\cite{shen2010examination}, exhibits pole to
pole oscillatory behavior in conjunction with the MinD protein and a
MinE that tends to be more localized in the center of the
cell~\cite{hu1999topological, fu2001mine, shapiro2009and, yu1999ftsz,
  raskin1999rapid}. The MinC protein will then have a higher time
averaged concentration in the cell poles, as opposed to the center
region of the cell, aiding in prohibiting the FtZ from developing in
the wrong region.

Studies have shown that the cell division process is capable of
preventing minicelling for fairly perturbed cell
shapes\cite{touhami2006temperature}. Varma et al. studied one
particular example of a perturbed shape, a three-pronged tube, and
found both experimental and simulation results\cite{varma2008min} that
show oscillation. These oscillations seem to seek out the extreme
poles in the cell shape\cite{corbin2002exploring}
\cite{juarez2010changes}, and effectively prevent minicelling in most
cases. However, Mannik et al. have also shown that there are
limitations to this mechanism\cite{mannik2010bacteria}
\cite{mannik2009bacterial}. One instance of this behavior is
observable when the cell has been forced into a flattened, irregular
shape. Oscillations in this sort of cell shape are far from the
regular sorts of oscillations seen in the computational work done by
Huang et al\cite{huang2003dynamic}. We seek to apply a computational
model to explore the more extreme or irregular cell shapes, in order
to determine the limits of the steady cell oscillations seen by both
theory and experiment.

A significant amount of work has been done to develop protein reaction
and diffusion models that exhibit accurate macroscopic dynamics of the
MinD protein system. Early models involve free proteins that affect
each others' rates of diffusion and membrane attachment, but do not
combine into compound states~\cite{meinhardt2001pattern}.  In 2003
Huang improved upon this work with a simple simulation model based on
MinD-MinE combination, ATPase hydrolysis, and MinD membrane attachment
that exhibits accurate MinD oscillations in
cylindrical cells~\cite{huang2003dynamic}. In this model cytoplasmic
MinD is more likely to attach to the membrane when MinD is already
clustered there (following observed non-linear attachement of minD on
the cell membrane), and is stationary once attached.  A number of
studies have used an approach similar in that they do not rely on the
ability of MinD to move along the walls and
cluster~\cite{kruse2007experimentalist, meinhardt2001pattern,
  drew2005polymerization, fange2006noise, kerr2006division}, while
studies have been made as well of models which rely on MinD mobility
and attraction on the cell membrane~\cite{kruse2002dynamic,
  howard2005cellular}.  Variations of the Huang 2003 model that
stochasticly account for variations of molecular interaction
\cite{fange2006noise} and monte-carlo simulations that implement this
stochastic version of Huang's mean field reaction rates confirm the
major results obtained by Huang's model, and more successfully predict
experimentally observed oscillations in round cell
phenotypes~\cite{drew2005polymerization, fange2006noise,
  huang2004min}. Biochemical models of broader scope have also been
used to study the MinD system and show consistent
results.\cite{arjunan2010new}.  However, in general, the results of
the stochastic and monte-carlo simulations are similar to those given
by Huang's mean field results.

Previous studies have been made of the Min system's associtation with
the cell membrane
\cite{hsieh2010direct}\cite{mileykovskaya2003effects}.  Studies have
shown as well that MinD binds preferentially to regions enriched with
cadiolipin, an anionic phospholipid that collects on regions of high
negative curvature. This mechanism has been incorporated into other
models.\cite{drew2005polymerization,cytrynbaum2007multistranded,renner2012mind,renner2012mind}
However, this mechanism of combined clustering, phospholipids and MinD
has not been observed in real cells. \cite{halatek2012highly}


\begin{figure}
  \includegraphics[width=\columnwidth]{reactions}
  \caption{Reactions included in the model of Huang \emph{et
      al.}~\cite{huang2003dynamic}.}\label{fig:reactions}
\end{figure}


\section{Model, Methods, and Pill Shaped Cell}
We implement the model of Huang \emph{et al.}  for the behavior of the
MinD and MinE proteins inside the cell~\cite{huang2003dynamic}.
Figure~\ref{fig:reactions} shows the reaction process.  The
cytoplasmic MinD:ADP complex undergoes nucleotide exchange and is
changed into the MinD:ATP complex.  This will naturally diffuse
towards and attach to the cell membrane.  A cytoplasmic MinE will
attach to the wall bound MinD:ATP complex and after a time will
activate ATP hydrolosis.  This breaks up the complex, releasing MinE,
phosphate, and MinD:ADP back into the cytoplasm.  The MinD:ADP will undergo
nucleotide exchange and begin again the cyclic process.  This model is
defined by a set of five reaction-diffusion equations:

\begin{multline}
  \frac{\partial \rho_{D:ADP}}{\partial t} = \mathcal{D}_D\nabla^2\rho_{D:ADP}-k_D^{ADP\rightarrow ATP}\rho_{D:ADP}\\
  +\delta(d_w)k_{de}\sigma_{DE},\hspace{3.4cm}
\end{multline}
\begin{multline}
  \frac{\partial \rho_{D:ATP}}{\partial t} = \mathcal{D}_D\nabla^2\rho_{D:ATP}+k_D^{ADP\rightarrow ATP}\rho_{D:ADP}\\
  -\delta(d_w)[k_D+k_{dD}(\sigma_D+\sigma_{DE})]\rho_{D:ATP}
\end{multline}
\begin{multline}
  \frac{\partial \rho_E}{\partial t} = \mathcal{D}_E\nabla^2\rho_E+\delta(d_w)k_{de}\sigma_{DE}
  -\delta(d_w)k_E \sigma_D \rho_E
\end{multline}
\begin{multline}
  \frac{\partial \sigma_D}{\partial t} = -k_E\sigma_D\rho_E
  +[k_D+k_{dD}(\sigma_D+\sigma_{DE})]\rho_{D:ATP}
  \label{eq:d-on-wall}
\end{multline}
\begin{multline}
  \frac{\partial \sigma_{DE}}{\partial t} = -k_{de}\sigma_{DE}+k_E\sigma_D\rho_E\hspace{3cm}
  \label{eq:FifthPDE}
\end{multline}

where $\rho$ is cytoplasmic protein density ($\mu m^{-3}$), $\sigma$
is membrane bound density ($\mu m^{-2}$), $\mathcal{D}_D$ and
$\mathcal{D}_{E}$ th diffusions constants for MinD and MinE,
respectively, $k_D^{\textrm{ADP $\rightarrow$ ATP}}$ the rate of
conversion from MinD:ADP to the MinD:ATP complex, $k_D$ the rate of
MinD:ATP attachement to the membrane when no protein is already
attached there, $k_{dD}$ the increase of this rate when MinD:ATP is
present on the membrane, $k_{de}$ the rate of hydrolisis of the
MinD:MinE:ATP complex, $k_E$ the rate of cytoplasmic MinE binding to
membrane bound MinD:ATP complex, and $d_w$ is the distance from the
point in space to the closest wall (which will always be
perpendicular to this distance).  $\delta(d_w)$ has units of
$\mu m^{-1}$ and will be zero everywhere except at the wall.
Equations \ref{eq:d-on-wall} and \ref{eq:FifthPDE} will only be
relevent at the membrane because the membrane bound density values
will have no meaning in the cytoplasm where there is no membrane.

Our diffusion and reaction rates are shown below.  We are interested
primarily in the effect of cellular size and shape on the protein
oscillations, so we follow Huang\cite{huang2003dynamic} and do not
deviate from the wild type values used in the cited work (see values
below).  Huang's simulations use total MinD and MinE concentrations of
$1,000/\mu m$ and $350/\mu m$, respectively, in a cylindrical cell of
radius $0.5\mu m$, and in our (non-cylindrical) cells we use the same
number of proteins per unit volume.  These concentration values are
1273 $\mu m^{-3}$ and 446 $\mu m^{-3}$, respectively.
\begin{gather*}
  \mathcal{D}_D = \mathcal{D}_{E} = 2.5\micron^2/\text{sec}\\
  k_D^{\textrm{ADP $\rightarrow$ ATP}} = 1/\textrm{sec,  }
  k_D = 0.025 \micron /\textrm{sec}\\
  k_{dD} = 0.0015 \micron^3/ \textrm{sec,  }
  k_{de} = 0.7/\textrm{sec}\\
  k_E = 0.093 \micron^3 /\textrm{sec}.
\end{gather*}
A 3D grid is then constructed in cartesian coordinates, with a grid
spacing of .05\micron.

In order to further our understanding of the results of various
simulation approaches to this system, we have performed both a basic
finite analysis, deterministic simulation that is spatially and
temporally discrete, and a stochastic analysis that is spatially
discrete but continuous in time.  Our stochastic simulation method
follows the work of Kraus~\cite{kraus1996crosstalk} which in turn
follows a method introduced by Gillespie~\cite{gillespie1977exact}.

We mean to investigate the geometric limits of the Min system
oscillations as observed by Mannik \emph{et
  al.}~\cite{mannik2012robustness}, so we model the Min system in a
variety of cell shapes and sizes.  Here we present a selection of
those, first naturally occuring pill-shaped cells and then a
number of flattened out shapes which reflect the experiments of Mannik
\emph{et al.}, in which bacteria are confined within a thin slit of
height $0.25\micron$ ~\cite{mannik2012robustness}. Viewed from the top
down the cells will have the shapes described below and viewed from
the side they have at their edges a semicircular protrusion (one may
imagine the edges of a pancake). Central to our observations are the
effects of cell size and so we show expanded cells that have the same
thickness but increased dimensions in the two planar directions.  We
present two shapes that are designed to replicate those published by
Mannik, equilateral and iscosolese triangle shapes, and a 'V' shape of
our own creation in order to best illustrate the effect of cell size
on oscillation stability.

%% Our pill shapes differ from those of Huang \emph{et al.} in that they
%% are cylinders with hemisphere endcaps instead of pure cylindrical
%% shapes.  Our cylinderidrical radius is $0.5\micron$ and the lengths
%% of our cells (measured between the tips of the endcaps) are
%% $5\micron$, $4\micron$, $3\micron$, and $2.5\micron$.

Kubitschek has shown in multiple experiments that at the time of cell
division cells have a volume that is within a range of roughly
$1\micron^3$ to $2\micron^3$~\cite{kubitschek1990cell,
  kubitschek1968linear}.  We follow Huang's
simulations\cite{huang2003dynamic} and Mannik's experiments and model
cells that are slightly larger than this range.

\begin{figure*}
  \includegraphics[width=\textwidth]{../data/shape-p/3_00-0_50-0_00-0_00-15_00-exact/plots/image-plot}
  \includegraphics[width=\textwidth]{../data/shape-p/3_00-0_50-0_00-0_00-15_00-full_array/plots/single-image-plot}
  \caption{Countour plot images of the concentration of each protein
    species in a natural pill-shaped bacterium at regular intervals in
    time (one second intervals), with darker regions indicating higher
    concentration. The upper plots shows results from the
    deterministic simulation and the lower shows results from the
    stochastic simulation.  The cells pictured are $4\micron$ in
    length, measured from end to end.  The order of frames is such
    that individual MinD proteins begin at the bottom of the plot (in
    the MinD:ATP state in the cytoplasm), and progress upward until
    they reach the MinE:MinD:ATP membrane-bound complex.  At that
    point, they will spontaneously dissociate into cytoplasmic MinE
    (the top row) and the starting state of cytoplasmic MinD:ADP.  The
    densities plotted are integrated along the axis orthogonal to the
    page, and the color scale is chosen separately for each species
    with black as the maximum value over space and time.  The
    stochastic densities are smeared out over space using a guassians
    approximation developed by Zhang et all~\cite{zhang2007gaussian}}.
  \label{image-p}
\end{figure*}

We begin with the naturally occuring pill cell shape.  We peice this
shape together as two hemispherical endcaps attached on either end of
a cylinder.  This shape follows the early simulations of Huang
\emph{et al.} but differs in that we have added the end caps for a
more natural shape, expecting similar results.  We test cells of
radius $.5\micron$ and of lengths $2\micron$, $3\micron$, and
$5\micron$ measured from the tip of one end cap to the tip of the
other. In each case, we observe in the deterministic simulations a
quick establishement (within a few periods) of extremely regular
oscillatory behavoir that lasts indefinitely.  The stochastic results
show oscillitions that are similarly regular in time, with periods
that nearly match those of the deterministic (they differ roughly by a
factor of .05). The stochastic results also show a slight variation in
the height of the density maxima, but the peaks are still very well
defined. The period increases with cell length, as shown in
Table~\ref{tab:pill-periods}, in agreement with the results of Huang
\emph{et al.}.  The stochastic simulations show a similar pattern.
The MinD maxima occur at the ends of the cell in a back and forth
manner, and with the same approximate period.

\begin{table}
  \begin{tabular}{|r|c|c|c|l|}
    \hline
    Length($\mu$) & 2.50 & 3.00 & 4.00 & 5.00\\
    \hline
    Period(sec) & sim & 33 & 38 & 48 \\ \cline{2-2}
    \hline
  \end{tabular}
  \caption{Period of oscillation according to length of cell for
    pill-shaped cells.}\label{tab:pill-periods}
\end{table}

This simple cell shape is a good starting point for observing in
detail the dynamic interaction between the different stages of protein
that lead to their qualitatively distinct behavoir. Figure
\ref{image-p} shows the process in a series of frames taken from
simulation animations for both the deterministic and stochastic
simulations.  We have 'smeared out' the stochastic concentrations in a
manner meant to reproduce the images shown by diffraction limited
flourescence microscopy.  We do so using the two dimensional gaussian
approximation developed by Zhang et all~\cite{zhang2007gaussian}.  In
this approximation we use a numerical aperture value of 1.3, which is
the same as used by mannik experimentally, and a wavelength of
$509nm$, typical of green flourescent microscopy.  Each frame is 2.5
seconds ahead of the last, and each image shows a the concentration of
a given state of protein summed over the coordinate orthogonal to the
page.  The color scale for each protein state is set according to the
maximum values observed.

At $t=0$ in Figs.~\ref{image-p}, the cytoplasmic MinE is concentrated
in the lower portion of the cell and is diffusing upward, since it
will react with and stick to wall-bound MinD that is concentrated on
the membrane in the upper portion of the cell.  At the boundary
between the cytoplasmic MinE and the membrane-bound MinD, there is a
ring of the membrane-bound MinE:MinD complex, which is moving towards
the upper end of the cell, removing the MinD from the membrane as it
goes.  The MinD is removed from the membrane in its MinD:ADP form and
diffuses freely for a time before spontenously undergowing nucleotide
exchange and shifting back into the MinD:ATP that is able to attach to
the membrane.  Diffusion away from the end of the cell will bring the
protein into the MinE ring, where reactions will prevent it from
collecting for any significant time on the membrane.  Diffusion
further towards the upper end, however, while the MinE has not yet
collected there, will allow MinD:ATP to accumulate with the other
MinD:ATP, peaking at around 10 seconds in Figure~\ref{image-p}.  As
the MinE ring approaches the end of the cell, seen in the seconds
following 10, the only freely diffusing MinD:ADP that 'escapes' the
MinE will be that which has diffused away from the the end of the cell
and past the MinE ring towards the center.

At 15 seconds, the membrane-bound MinD:ATP has been essentially
removed from the top end of the cell, and MinD:ATP has begun to bind
to the membrane in the the lower half of the cell.  At this stage,
there is a high concentration of cytoplasmic MinE at the top of the
cell, and by 16 seconds we begin to see the formation of a MinE ring
just below the center of the cell.  At 18 seconds, the cell has
reached its original state (reversed directionally), with MinD:ATP
bound to the membrane on the lower third of the cell, a high
concentration of cytoplasmic MinE in the upper half of the cell, and a
MinE ring pushing downward on the membrane-bound MinD:ATP.

\section{Stability Analysis}
\label{sec:stability-analysis}
We have found that in both methods of simulation there is a lower
length scale that characterizes much of the behavoir.  Regular
oscillatory behavoir between two poles at opposite ends of the longer
cellular direction is found in cells which are long enough in one
dimension for the reaction chain to be unstable but short enough in
the perpendicular dimension to be stable.  The limits are seen when
both dimensions are too short to be unstable, so that initial
assymetry in protein concentration relaxes into a state of no
oscillation.  In these cell sizes, the deterministic simulations show
a motionless steady state and the stochastic simulations show random
fluctuations without any bulk behavoir (see Figure~\ref{box-mannik}).

In order to explain this theoretically, we have performed stability
analysis of the five differential equations in an infinite slab that
show a characteristic half wavelength in perturbative spatial
oscillations at which the system becomes unstable.  This length is
dependent upon slab thickness as shown in
Figure~\ref{fig:stability-analysis}.  When increasing the lengths and
widths of our simulated flattened cells, the cells stop exhibiting two
pole oscillalitory behavior and instead exhibit multiple positions of
maximization when the shortest distance across the cell is
approximately equal to the characteristic half wavelength at the
particular thickness at which we simulate (our simulations of cells
with thickness $0.25\mu m$ corresponds to half wavelengths of $2.13\mu
m$).  Huang~\cite{huang2003dynamic} performs a similar analysis for a
cylindrical shape \fixme{is it cylindrical or 1d?} and finds
characteristic half wavelengths of $2\mu m$.


Decreasing the cell size in succesive steps, we can see the critical
size at which the cell oscillations are quickly damped out.
Figure~\ref{box-mannik} shows box plots of the largest cells simulated
whose oscillations relax into the steady state solution.  Both have a
long axis of roughly 2.2$\mu m$, in agreement with theory.

\section{Comparison of Mannik's Experimental and Our Stadium Shapes}
\fixme{should I say anything about different/large sizes?}  We have
made movies out of color plots of MinD concentration, integrated along
the flattened axis, in which each frame is seperated in simulation
time by a half second. Our simulations include the two flattened cell
shapes that are reported in Manniks paper, a few other 'irregular'
shapes of our own creation, triangle shaped flattened cells, and a
symetric 'stadium' shaped cell. For each of these shapes we have
varied the size, from roughly twice the dimensions of Mannik's
reported cells down to the smaller sizes refered to in
section~\ref{sec:stability-analysis}.

We have found that the deterministic simulations of the protein
reaction sequence discussed above results in robust oscillatory
behavoir of polar selection and oscillation in not only symmetrical
shapes, such as the wildtype pill shape, but also in very
assymetrical, flatttened cells, such as those studied by Mannik.
After the lower stability limit of
section~\ref{sec:stability-analysis} has been surpassed, this
qualitative behavoir doesn't seem to have size dependence until cells
reach sizes that have more than twice the dimensions of the Mannik
cells.  At this point the robust oscillations, while still evident,
begin to lose their consistanty.  The Mannik reported cells already
seem to be rather large (wild type cells are typically smaller than
those observed my Mannik
\cite{kubitschek1990cell,kubitschek1968linear,mannik2012robustness}),
so we left off simulating cells any larger than twice this size. The
movies show sharp peaks of MinD in specific locatations, typically at
the edges of the cell and in regions of high radius of curvature,
followed by a spreading out of the MinD out into the cell, with no
real bias towards any location, until it aggregates again in a
specific location at the opposite edge of the cell, maximizing there.
The pattern is repeated again and again with surprising consistency,
which can be seen on the right of Figure~\ref{box-mannik}.  This is
the same plotting method as our box plots above, here showing
stochastic and deterministic simulations of cells that are the same
shape and size of one of Mannik's reported cells.

\begin{figure}
  \includegraphics[width=\columnwidth]{../data/shape-randst/0_25-18_50-18_50-95_00-15_00-full_array/plots/box-plot_D}
  \includegraphics[width=\columnwidth]{../data/shape-randst/0_25-18_50-18_50-95_00-15_00-exact/plots/box-plot_D}
  \caption{A comparison of our box plots that show a disparity between
    the behavoir of the stochastic and deterministic simulations in
    our Mannik cells.  The deterministic show a regular, robust
    oscillatory behavoir between the two ends of the cell. The
    stochastic simulation do show a general back and forth tendency
    between two ends of the cell, but it is not robust and tends to
    break down.}
  \label{box-mannik}
\end{figure}
\begin{figure}
  \includegraphics[width=\columnwidth]{../data/shape-randst/0_25-18_60-28_60-94_00-15_00-full_array/plots/box-plot_D}
  \includegraphics[width=\columnwidth]{../data/shape-randst/0_25-18_60-28_60-94_00-15_00-exact/plots/box-plot_D}
  \caption{A comparison of our box plots that show a disparity between
    the behavoir of the stochastic and deterministic simulations in
    our Mannik cells.  The deterministic show a regular, robust
    oscillatory behavoir between the two ends of the cell. The
    stochastic simulation do show a general back and forth tendency
    between two ends of the cell, but it is not robust and tends to
    break down.}
  \label{box-mannik}
\end{figure}

The results of these simulations differ significantly with Mannik's
experimental results that show much less regular behavoir, with a
large variation in the location of maxima.

On the other hand, our stochastic simulations show both robust pole to
pole oscillations in the symmetric cells and irregular behavoir in the
assymetrical shapes, agreeing well with Mannik's experimental
findings.

Figure~\ref{arrow-pill-stadium} and Figure~\ref{randst-plot-ave} show
color plots of time-averaged MinD overlayed by arrows depicting
sequential maxima that are global in space and local in time.  Our
stochastic method movies (and arrow plots of
Figure~\ref{randst-plot-ave}) show irregular location of maxima that
is similar to what is observed experimentally by Mannik. There is a
general bias towards back and forth behavoir, which can be seen in the
box plots of Figure~\ref{box-mannik}, but the behavoir is irregular
and the location of the maxima vary a significant amount.  On the
other hand, the deterministic simulations show behavoir that is much
too regular when compared with experiment, exhibiting locations of
maxima that do not vary.  The deterministic method is seen to be
inadequate.

\begin{figure*}
  \centering
  \includegraphics[width=\textwidth]{../paper/plot-ave}
  \caption{\fixme{Look into unifying the color scale, and adding a
      scale bar.} Arrows depicting global maxima in space and local
    maxima in
    time overlayed on color plots of time averaged total MinD density.
    A density threshold is chosen and only maxima above this threshold
    are shown. The plots serve to illistrate what is observed in our
    movies.  The simulation time covered in these is 1000 seconds
    \fixme{this is not true yet - need to change the data once its
      finished right now 2 of them are 1000 the others are like 980}.
    The top row shows plots published by Mannik of the MinD maxima
    behavoir and the bottom two rows show our pancake shapes that are
    designed to have the same shape.  The middle row shows results
    from the stochastic simulation and the bottom shows results from
    the deterministic.  We can see that the stochastic method shows
    behavoir that is similar to experiment while the deterministic
    method shows maximization that is too regular and robust.}
  \label{randst-plot-ave}
\end{figure*}

Judging that at this point that we have a fair idea of how our two
methods compare with experiment, we investigate the limits of the
break down of regular activity, by simulating large, regular,
'stadium' shapes.  These shapes have the flattened, pancake like form
of the assymetric cells, but from the top down they look as pills
do. In two dimensions they are composed of a rectangle with two half
ellipse connected on either end.  They are also made to have a large
volume that is similar to the Mannik shapes.

We find that both the stochastic and the deterministic solutions show
regular back and forth oscillations in the stadium shaped cells, as
can be seen Figure~\ref{box-stadium}.  This suggests that it is the
irregular, assymetric shape of the Mannik cells, not their size or
their flattened nature, that causes these cells to exhibit
irregularity in MinD oscillations.

\begin{figure}
  \includegraphics[width=\columnwidth]{../data/shape-stad/0_25-5_50-1_00-0_00-15_00-full_array/plots/box-plot_D}
  \includegraphics[width=\columnwidth]{../data/shape-stad/0_25-5_50-1_00-0_00-15_00-exact/plots/box-plot_D}
  \caption{A comparison of our box plots that show the similarity
    between the stadium shapes simulated stochastically and
    deterministically.  Both show regular oscillatory maxima with
    similar periods.  This is in contrast to our boxpplot comparison
    of one of the Mannik shapes in Figure~\ref{box-mannik}}
  \label{box-stadium}
\end{figure}


\fixme{specifically design other arrow plot (with many comparisons) to
  illustrate point}


Figure~\ref{arrow-compare-mannik} shows plots of total MinD
concentration maxima that are global in space and local in time.  The
tip of an arrow represents one of these maxima and the adjacent arrow
(whose tail is touching the previous arrow's tip) points to the next
such maxima in time.  Mannik \emph{et al.} have published similar
plots of two of their cells, and we have attempted to replicate the
shape of these cells in our simulation in order to compare.  We find
that when we replicate the size as well as shape of their cells as
near as possible, we do not see the disordered protein behavoir that
they observe.  Instead we see a robust ability of the cell to find two
prefered pole locations and to oscillate maxima back and forth very
steadily between them.  When we increase the size of our cells, we
indeed see multiple positions of maxima develop, but we never observe
the random behavoir that they have observed experimentally.  This is
perhaps due to the exact, continuous method we use when conducting
simulations.  Stochastic methods would perhaps show more random
behavoir at these sizes.  We consider this to be a limitation of the
exact solution method.



\section{Interpretation of Data}
\section{Conclusion}
No conclusion :(
\bibliography{paper}



\end{document}

\section*{Appendix}

\section{Below are NOTES for the Writing}


\end{document}
