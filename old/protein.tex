\documentclass[pdftex,10pt,a4paper]{article}
\usepackage[latin1]{inputenc}
\usepackage{amsmath}
\usepackage{amsfonts}
\usepackage{amssymb}
\usepackage{fullpage}
\usepackage{graphicx}
\usepackage{wrapfig}
\begin{document}
\bibliographystyle{unsrt}

commit this!!
Bump.fdsfdsfdsoblem  Hello cruel cruel World!  The other thing
It is vital during the process of bacterial cell division for proper nucleiod occlusion that the FtsZ polymer chain develop on the cell wall in the center of the cell.  Huang et all have shown in simulation that a system of Min protein interaction within the cell will lead to a natural oscillation of the MinC protein from one cell end to the other that will leave the center free of any build up of MinC.  This will allow for the PtsZ chain to develop in the center and not at the ends, where the two nucleiods are housed.  The cell has a built in method of ensuring that minicelling during cell division deos not occur.  The interaction takes place between MinE,MinD, and MinC, which naturally associates with MinD.  During an oscillation period, MinD accumulates at the end of the cell in a polar zone.  It is membrane-associated in its ATP bound form, and while membrane associated it recruits MinC to the cell wall (blocking FtsZ formation).  Meanwhile the MinE forms a cytoplasmic ring at the edge of the MinD polar zone with some MinE in the zone itself.  The MinD zone shrinks and builds up on the other side of the cell.  Huang's simulation involves basic interactions between the MinD, MinE, and ATP.  MinD:ATP builds up on the membrane, where it interacts with cytoplasmic MinE that activates hydrolisis and MinE and MinD:ADP enter back into the cell.

Mannik et all have shown that the formation of irregular cell shapes adversly effects the Min system's ability to maintain their regular oscillatory behavior, while the accuracy of daughter cell size in cell division is much more stable.  This shows the importance of the other regulating factor in cell division - nucleiod occlusion.  They use microfabricated silicon chips that have etched microchambers and shallow channels.  The debth of the chambers is 1.8$\mu$m and squeezes the bacteria into channels of .25$\mu$m.  Upon entering the channels, the cells quickly widen along their short axis by 30-40 percent and then slow remodeling of the cell wall with the mechanical stress, broadens the cells so that they can reach diameters of 5 $\mu$m.  Daighter cell size after division is recorded and the data is evaluated in histograms against the ratio of volume between the two cells.  Suprisingly, the central peak around a .5 division - so equal sizes between them, is largely not affected by the flattening of cells.  99 percent of the unflattened cells and 90 percent of the flattened ones fall in limits of .375 and .625, and the standard deviation of the peaks themselves are very similar.  With the flattened cells, the the ones that fall outside of this mid range create small spikes at .25 and .75, correspnding to daughter cells that have a 1:3 ratio.

They gneetically mutate the bacterial cells in order to eliminate certain proteins from them.  They eliminate MinC and SlmA.  SlmA deletion has very little effect accept for some assymetry in cell division area.  The MinC deletion shows a difference in volume division.  Only 73 percent divide with the approximate 1:1 ratio while for un deleted 99 percent do.  This effect is seen in both flattened and unflattened cells, and its very similar in both, which is suprising.
In the flattened cells, there is still MinD maxima of localization, but these will sometimes split into multiple maxima as opposed to just the the usual back and forth single maxima.  This causes the usual polarized time averages to be more uniform, which will cause the min protein system to have less of an effect on the cell division process.  
The nuclioed occlusion process works as, the nucleioeds seperate into the different poles, and the FtsZ begins to set up where the nuclioeds are not -into the gap in the center.  Mannik shows with time-lapse imaging that the gap between develops first and then the FtsZ.  Even in flattened and abberent shaped cells, there's a strong anti-correlation between FtsZ and chromosomes in the nucleioeds.  Most of this is about the occlusion of the nuclioeds by analyzing the anticorrelation between the FtsZ and the chromosomal mass.  So these anticorrelations are not effected by the deletion of MinC or slmA except for the FtsZ formation that leads to minicells.  In these cases, the anticorreltion is not strong, so the nucleiods seem to be closer and no occlusion (this is not totally it, it's top of pg 5 in this paper).  

Hello cruel, cruel world. \cite{huang2003dynamic} \cite{mannik2012robustness}
\cite{huang2006curvature}
\bibliography{protein}
\end{document}
